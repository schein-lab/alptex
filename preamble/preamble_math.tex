% !TEX root = main.tex

% MATH PACKAGES
% Note: amsfonts conflicts with mathdesign (loaded in preamble.tex)
% mathdesign provides its own blackboard bold, so we skip amsfonts
\usepackage{centernot}
\usepackage{amsthm}         % theorem environments
\usepackage{nicefrac}       % compact symbols for 1/2, etc.
\usepackage{mathtools}      % extends amsmath
\usepackage{amsbsy}         % bold math symbols
\usepackage{amstext}        % text in math mode
\usepackage{thmtools}       % theorem tools
\usepackage{thm-restate}    % restatable theorems

% compatibility w/ parskip https://tex.stackexchange.com/questions/25346/wrong-spacing-before-theorem-environment-amsthm
\begingroup
    \makeatletter
    \@for\theoremstyle:=definition,remark,plain\do{%
        \expandafter\g@addto@macro\csname th@\theoremstyle\endcsname{%
            \addtolength\thm@preskip\parskip
            }%
        }
\endgroup

\DeclareRobustCommand{\mb}[1]{\ensuremath{\mathbf{\boldsymbol{#1}}}}
% \DeclareRobustCommand{\mb}[1]{\mathbold{#1}}

\DeclareRobustCommand{\KL}[2]{\ensuremath{\textrm{KL}\left(#1\;\|\;#2\right)}}

% \DeclareMathOperator*{\argmax}{arg\,max}
% \DeclareMathOperator*{\argmin}{arg\,min}

\crefname{lemma}{lemma}{lemmas}
\Crefname{lemma}{Lemma}{Lemmas}
\crefname{thm}{theorem}{theorems}
\Crefname{thm}{Theorem}{Theorems}
\crefname{prop}{proposition}{propositions}
\Crefname{prop}{Proposition}{Propositions}
\crefname{assumption}{assumption}{assumptions}
\crefname{assumption}{Assumption}{Assumptions}


% \newtheorem{thm}{Theorem} % reset theorem numbering for each chapter
% \newtheorem{defn}{Definition} % definition numbers are dependent on theorem numbers
% \newtheorem{prop}[thm]{Proposition}
% \newtheorem{exmp}[thm]{Example} % same for example numbers
% \newtheorem{lemma}[thm]{Lemma}
% \newtheorem{assumption}{Assumption}
% \newtheorem{corollary}[thm]{Corollary}
% Independence symbol (double perp): use \indpt
\def\independenT#1#2{\mathrel{\rlap{$#1#2$}\mkern2mu{#1#2}}}



\def\checkmark{\tikz\fill[scale=0.4](0,.35) -- (.25,0) -- (1,.7) -- (.25,.15) -- cycle;} 

% Note: booktabs loaded in preamble.tex; arydshln adds dashed lines
\usepackage{arydshln}
\makeatletter
\def\adl@drawiv#1#2#3{%
        \hskip.5\tabcolsep
        \xleaders#3{#2.5\@tempdimb #1{1}#2.5\@tempdimb}%
                #2\z@ plus1fil minus1fil\relax
        \hskip.5\tabcolsep}
\newcommand{\cdashlinelr}[1]{%
  \noalign{\vskip\aboverulesep
           \global\let\@dashdrawstore\adl@draw
           \global\let\adl@draw\adl@drawiv}
  \cdashline{#1}
  \noalign{\global\let\adl@draw\@dashdrawstore
           \vskip\belowrulesep}}
\makeatother

\newenvironment{proofsk}{%
  \renewcommand{\proofname}{Proof sketch}\proof}{\endproof}

\renewcommand{\epsilon}{\varepsilon}

%********************************************************************
% Extra theorem environments
%********************************************************************

\declaretheorem[style=plain,numberwithin=section,name=Theorem]{theorem}
\declaretheorem[style=plain,sibling=theorem,name=Lemma]{lemma}
\declaretheorem[style=plain,sibling=theorem,name=Proposition]{proposition}
\declaretheorem[style=plain,sibling=theorem,name=Corollary]{cor}
\declaretheorem[style=plain,sibling=theorem,name=Claim]{claim}
\declaretheorem[style=plain,sibling=theorem,name=Conjecture]{conj}
\declaretheorem[style=definition,sibling=theorem,name=Definition]{defn}
\declaretheorem[style=definition,name=Assumption]{assumption}
\declaretheorem[style=definition,sibling=theorem,name=Example]{example}
\declaretheorem[style=remark,sibling=theorem,name=Remark]{remark}

\newenvironment{example*}
 {\pushQED{\qed}\example}
 {\popQED\endexample}
\numberwithin{equation}{section}
